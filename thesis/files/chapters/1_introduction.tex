\chapter{Introduzione}
\label{chap:introduzione}
\section{L'azienda}
Spazio Dev S.r.l.\footcite{site:spazio-dev} è una software house situata a Tombolo (PD) fondata da due soci, ad oggi l'azienda conta circa 17 dipendenti e si occupa di offrire servizi legati al mondo dello sviluppo web e dell'ottimizzazione dei processi industriali.
L'azienda si occupa, nello specifico, di sviluppare siti web, e-commerce, software su misura e di integrare algoritmi che sfruttano l'intelligenza artificiale per ottimizzare la produzione o monitorare lo stato dell'azienda in tempo reale.
Il logo aziendale è mostrato nella figura \ref{fig:logo-spazio-dev}.
\begin{figure}[H]
	\centering
	\includegraphics[alt={Logo dell'azienda Spazio Dev}, width=1\columnwidth]{img/logo_spazio_dev.jpg}
	\caption{Logo dell'azienda Spazio Dev}
	\label{fig:logo-spazio-dev}
\end{figure}

\subsection{Mugalab}
Mugalab\footcite{site:mugalab}, il cui logo è mostrato nella figura \ref{fig:logo-mugalab}, è la divisione di Spazio Dev S.r.l.\ dedicata alla gestione della sala stampa 3D presente nella sede aziendale.
Attraverso la tecnica della stampa additiva, Mugalab si occupa di:
\begin{itemize}
  \item progettare e prototipare componenti per il settore industriale;
  \item progettare e prototipare componenti per dispositivi elettronici;
  \item disegnare e produrre oggetti ornamentali.
\end{itemize}

\begin{figure}[H]
  \centering
  \includegraphics[alt={Logo della divisione Mugalab}, width=0.45\textwidth]{img/logo_mugalab.png}
  \caption{Logo di Mugalab}
  \label{fig:logo-mugalab}
\end{figure}

\section{Il progetto}
\subsection{L'idea}
L'idea di questo percorso di stage curricolare nasce dalla necessità dell'azienda di gestire in maniera efficiente la produzione delle componenti stampate in 3D.
Attualmente lo stabilimento dispone di 12 stampanti e la pianificazione della stampa degli ordini di produzione viene effettuata manualmente da un singolo operatore,
il quale si accerta di ottimizzare la produzione seguendo alcune buone pratiche che permettono di risparmiare tempo e aumentarne l'efficienza.
L'obiettivo principale dell'azienda è quindi quello di ideare un sistema automatizzato, il cui scopo è creare una coda di stampa ottimizzata a partire dagli ordini di produzione
non ancora evasi completamente.
Tale sistema dovrà integrarsi completamente con il sistema gestionale esistente, il quale viene utilizzato quotidianamente dal personale dell'azienda per gestire la coda di stampa e registrare gli ordini
ricevuti dalle piattaforme utilizzate per la vendita delle componenti e degli oggetti ornamentali (Amazon, Shopify e vendita B2B).
Questo progetto è destinato esclusivamente ad uso interno dell'azienda, assumendo quindi il ruolo di \gls{pocg} della fattibilità di un sistema di schedulazione automatizzato e vincolato.
Il codice potrà quindi essere ulteriormente ottimizzato e ampliato in modo tale da migliorarne l'efficienza e aumentare la qualità del risultato prodotto, oltre ad essere adattato a scenari d'utilizzo differenti
da quello proposto.

\subsection{Motivazioni e problematiche attuali}
La pianificazione manuale della coda di stampa genera ritardi e inefficienze operative, con ricadute dirette sui tempi di consegna.
Una schedulazione basata su un algoritmo di ottimizzazione consente di generare piani fattibili in modo più rapido e meno incline a errori, migliorando l'utilizzo delle stampanti e riducendo i tempi morti. 

\section{Organizzazione del testo}
\begin{description}
    \item[{\hyperref[chap:contesto-aziendale]{Il secondo capitolo}}] descrive il contesto aziendale, i processi di sviluppo software e la gestione della sala stampe.
    
    \item[{\hyperref[chap:descrizione-stage]{Il terzo capitolo}}] descrive il progetto di stage, l'analisi preventiva dei rischi e la pianificazione.
    
    \item[{\hyperref[chap:analisi-requisiti]{Il quarto capitolo}}] contiene l'analisi dei requisiti del progetto.
    
    \item[{\hyperref[chap:progettazione-codifica]{Il quinto capitolo}}] descrive la fase di codifica, progettazione e validazione del prodotto.
        
    \item[{\hyperref[chap:conclusioni]{Il sesto capitolo}}] contiene un'analisi critica del prodotto.
\end{description}

Durante la stesura del testo sono state adottate le seguenti convenzioni tipografiche:
\begin{itemize}
	\item gli acronimi, le abbreviazioni e i termini ambigui o di uso non comune menzionati vengono definiti nel glossario, situato alla fine del presente documento;
	\item per la prima occorrenza dei termini riportati nel glossario viene utilizzata la seguente nomenclatura: \gls{apig};
	\item i nomi degli oggetti, delle funzioni, delle tabelle, delle colonne, delle classi e delle rotte \textit{API} sono evidenziati con il carattere \texttt{monospaziato}.
	\item i termini in lingua straniera o facenti parte del gergo tecnico sono evidenziati con il carattere \textit{corsivo}.
\end{itemize}

\newpage