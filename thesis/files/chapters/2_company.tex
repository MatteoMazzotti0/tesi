\chapter{Il contesto aziendale}
\label{chap:contesto-aziendale}
\section{Sviluppo software}
Spazio Dev è un'azienda di piccole dimensioni, di recente fondazione e ancora in fase di crescita. Di conseguenza, il progetto è stato sviluppato con budget limitato e da una singola persona.
I processi di sviluppo aziendali, pertanto, sono ancora in rapida evoluzione.
\subsection{Version Control System}
L'azienda utilizza Git\footcite{site:git} come \gls{VersionControlSystemg} e Gitea\footcite{site:gitea} come piattaforma di hosting per i repository.
Per quanto riguarda il \gls{GitBranchingModelg} viene adottato il Gitflow workflow\footcite{site:gitflow-workflow}, adattato nel seguente modo:
\begin{itemize}
    \item il \textit{branch} dedicato alle versioni stabili è denominato \textit{main};
    \item il \textit{branch} dedicato alle versioni di sviluppo è denominato \textit{develop};
    \item i \textit{branch} dedicati alle implementazioni di \gls{featureg} vengono denominati con lo schema \textit{feature/<nome\_feature>} dove il parametro \textit{<nome\_feature>} è una breve descrizione della \textit{feature} che si sta codificando;
    \item i \textit{branch} dedicati ai \gls{bugfixg} vengono denominati con lo schema \textit{fix/<nome\_bug>} dove il parametro \textit{<nome\_bug>} è una breve descrizione del \gls{bugg} che si sta risolvendo.
    \item i \textit{branch} dedicati alla documentazione vengono denominati con lo schema \textit{docs/<nome\_documentazione>} dove il parametro \textit{<nome\_documentazione>} è una breve descrizione della modifica alla documentazione.
\end{itemize}

\begin{figure}[H]
  \centering
  \includegraphics[alt={Schema del funzionamento di Gitflow Workflow}, width=\textwidth]{img/gitflow.png}
  \caption{Funzionamento Gitflow Workflow}
  \label{fig:gitflow}
\end{figure}

Viene inoltre adottato lo standard Conventional Commits 1.0.0\footcite{site:conventional-commits} per la scrittura dei messaggi di commit, in particolare
ogni commit possiede un messaggio così formato: \textit{<tipo>: <descrizione>}, dove il parametro
\textit{<tipo>} è:
\begin{itemize}
    \item \textit{fix} in caso di commit contenente \textit{bugfix};
    \item \textit{docs} in caso di commit di aggiornamento della documentazione;
    \item \textit{feat} in caso di commit contenente \textit{feature}.
\end{itemize}
Non sono state definite policy di protezione dei rami, requisiti di revisione o regole di merge, tali aspetti potranno tuttavia essere definiti in futuro dall'azienda.

\subsection{Modello di lavoro agile}
L'azienda adotta il framework Scrum\footcite{site:scrum-framework} come modello di lavoro agile\footcite{site:agile-manifesto}, in particolare:
\begin{itemize}
    \item il ruolo di \gls{productOwnerg} viene ricoperto dai titolari dell'azienda i quali si occupano di creare e gestire il \gls{backlogg} del prodotto, fornire indicazioni al team su quali feature implementare e decidono le scadenze dei rilasci del prodotto;
    \item il ruolo di \gls{scrumMasterg} viene ricoperto dai titolari dell'azienda, i quali si occupano di pianificare \gls{sprintPlanningg}, \gls{dailyScrumg}, \gls{sprintReviewg} e \gls{sprintRetrospectiveg};
    \item il ruolo del team di sviluppo viene ricoperto dagli sviluppatori, i quali vengono divisi in gruppi e assegnati ai progetti in fase di sviluppo.
\end{itemize}
\begin{figure}[H]
  \centering
  \includegraphics[alt={Schema del funzionamento del framework Scrum}, width=\textwidth]{img/scrum.png}
  \caption{Framework Scrum}
  \label{fig:scrum}
\end{figure}

\subsection{Issue Tracking System}
L'azienda utilizza Plane\footcite{site:plane} come sistema di tracciamento delle attività. 
Le \gls{issueg} sono organizzate su una board in stile Kanban composta da cinque colonne:
\begin{itemize}
    \item \textit{Backlog}: raccolta iniziale delle attività proposte, non ancora pianificate;
    \item \textit{Todo}: attività prioritarie selezionate per l'esecuzione;
    \item \textit{In Progress}: attività in lavorazione;
    \item \textit{In Review}: attività concluse e in fase di verifica;
    \item \textit{Done}: attività completate.
\end{itemize}

Ogni \textit{issue} è inoltre associata a un livello di priorità tra i seguenti:
\begin{itemize}
    \item \textit{Critical}: attività richiedenti intervento immediato;
    \item \textit{High}: richieste ad alto impatto o con scadenza molto ravvicinata;
    \item \textit{Medium}: \textit {bugfix} non critici che vengono inseriti nella pianificazione ordinaria degli \gls{sprintg};
    \item \textit{Low}: miglioramenti minori non prioritari.
\end{itemize}
La combinazione di colonna Kanban e priorità rende evidente sia lo stato di avanzamento sia l'urgenza.

\begin{figure}[H]
  \centering
  \includegraphics[alt={Rappresentazione grafica della Kanban board}, width=\textwidth]{img/kanban.png}
  \caption{Kanban board}
  \label{fig:kanban}
\end{figure}

\subsection{Strumenti di comunicazione}
Le comunicazioni interne avvengono interamente tramite Telegram\footcite{site:telegram}: 
ogni progetto dispone di un gruppo dedicato in cui titolari e sviluppatori condividono aggiornamenti, 
documentazione e decisioni operative in tempo reale. Questo canale unico permette di evitare dispersioni e di mantenere 
traccia delle richieste provenienti dai clienti. 

Le riunioni operative vengono organizzate con cadenza variabile, 
in funzione delle necessità del progetto o delle scadenze concordate con i clienti. 
Quando emergono nuove priorità o si avvicina un rilascio, i titolari convocano incontri mirati per allineare il team sugli obiettivi, 
discutere eventuali impedimenti e definire le attività da pianificare nel successivo \textit{sprint}.

\section{Gestione della sala stampe}
\label{section:printing-management}
Spazio Dev possiede attualmente molteplici punti di ingresso degli ordini relativi alla stampa 3D dei componenti, in particolare:
\begin{itemize}
  \item La vendita di vasi portafiori e complementi d'arredo viene effettuata online e sulla piattaforma e-commerce Amazon;
  \item Le produzioni personalizzate e B2B vengono gestite manualmente.
\end{itemize}
Per la gestione di tali ordini e della coda di stampa viene utilizzato un gestionale sviluppato dall'azienda stessa, denominato "Idrotech Manager".
\subsection{Ottimizzazione della coda di stampa}
Per gestire in maniera efficiente l'evasione degli ordini l'azienda applica diverse ottimizzazioni alla coda di stampa, in modo tale da sfruttare nel miglior modo
possibile le capacità di produzione.
Tali procedure riguardano in particolare gli orari di avvio della stampa, la gestione dei cambi di materiale e dei cambi di ugello.
\subsubsection{Orari di avvio della stampa}
La gestione degli orari di avvio della stampa risulta fondamentale per ottimizzare al meglio la produzione.
Ciò è dovuto al fatto che ogni stampante, quando termina la produzione di un piatto, necessita di un intervento manuale da parte di un operatore per la rimozione
del prodotto finito e l'avvio della stampa successiva.
L'idea alla base è quindi quella di pianificare tutte le stampe brevi durante l'orario lavorativo (in modo che sia sempre possibile avviare una nuova produzione).
Le stampe più lunghe vengono invece avviate poco prima della fine della giornata per sfruttare al meglio i periodi di tempo in cui non c'è personale presente all'interno dello stabilimento.
\subsubsection{Cambi di materiale}
La coda di stampa può essere ottimizzata riducendo al minimo i cambi di materiale (o colore) tra una stampa e quella successiva.
Questo perché ogni qual volta viene richiesto di cambiare il colore del filamento o il materiale utilizzato la stampa viene interrotta e un operatore deve intervenire manualmente
per la sostituzione del filamento.
Per semplicità, l'azienda assume che il tempo necessario per cambiare un filamento sia di 5 minuti.
\subsubsection{Cambi ugello}
Ogni articolo possiede uno specifico ugello che viene utilizzato per cambiare lo spessore della stampa.
Come per i cambi di materiale anche il cambio ugello richiede un intervento manuale da parte di un operatore.
Per semplicità, l'azienda assume che il tempo necessario per cambiare un ugello sia di 15 minuti.
\newpage
