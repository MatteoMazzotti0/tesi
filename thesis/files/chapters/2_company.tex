\chapter{Il contesto aziendale}
\label{chap:contesto-aziendale}
\section{Metodologia aziendale}
Spazio Dev è un'azienda di piccole dimensioni, di recente fondazione e ancora in fase di crescita, di conseguenza il progetto è stato sviluppato con budget limitato e da una singola persona.
I processi di sviluppo aziendali, pertanto, sono ancora in rapida evoluzione.
\subsection{Sviluppo software}
\subsubsection{Version Control System}
L'azienda utilizza Git\footcite{site:git} come \gls{VersionControlSystemg} e Gitea\footcite{site:gitea} come piattaforma di hosting per i repository.
Per quanto riguarda il \gls{GitBranchingModelg} viene adottato il Gitflow workflow\footcite{site:gitflow-workflow}, adattato nel seguente modo:
\begin{itemize}
    \item il \textit{branch} dedicato alle versioni stabili è denominato \textit{main};
    \item il \textit{branch} dedicato alle versioni di sviluppo è denominato \textit{develop};
    \item i \textit{branch} dedicati alle implementazioni di \gls{featureg} vengono denominati con lo schema \textit{feature/<nome\_feature>} dove il parametro \textit{<nome\_feature>} è una breve descrizione della \textit{feature} che si sta codificando;
    \item i \textit{branch} dedicati ai \gls{bugfixg} vengono denominati con lo schema \textit{fix/<nome\_bug>} dove il parametro \textit{<nome\_bug>} è una breve descrizione del \gls{bugg} che si sta risolvendo.
    \item i \textit{branch} dedicati alla documentazione vengono denominati con lo schema \textit{docs/<nome\_documentazione>} dove il parametro \textit{<nome\_documentazione>} è una breve descrizione della modifica alla documentazione.
\end{itemize}

\begin{figure}[H]
  \centering
  \includesvg[width=0.9\textwidth]{gitflow}
  \caption{Funzionamento Gitflow Workflow}
  \label{fig:gitflow}
\end{figure}

Viene inoltre adottato lo standard Conventional Commits 1.0.0\footcite{site:conventional-commits} per la scrittura dei messaggi di commit, in particolare
ogni commit possiede un messaggio così formato: \textit{<tipo>: <descrizione>}, dove il parametro
\textit{<tipo>} è:
\begin{itemize}
    \item \textit{fix} in caso di commit contenente \textit{bugfix};
    \item \textit{docs} in caso di commit di aggiornamento della documentazione;
    \item \textit{feat} in caso di commit contenente \textit{feature}.
\end{itemize}
Non sono state definite policy di protezione dei rami, requisiti di revisione o regole di merge, tali aspetti potranno tuttavia essere definiti in futuro dall'azienda.

\subsubsection{Modello di lavoro agile}
L'azienda adotta il framework Scrum\footcite{site:scrum-framework} come modello di lavoro agile\footcite{site:agile-manifesto}, in particolare:
\begin{itemize}
    \item Il ruolo di \gls{productOwnerg} viene ricoperto dai titolari dell'azienda i quali si occupano di creare e gestire il \gls{backlogg} del prodotto, fornire indicazioni al team su quali feature implementare e decidono le scadenze dei rilasci del prodotto;
    \item Il ruolo di \gls{scrumMasterg} viene ricoperto dai titolari dell'azienda, i quali si occupano di pianificare \gls{sprintPlanningg}, \gls{dailyScrumg}, \gls{sprintReviewg} e \gls{sprintRetrospectiveg};
    \item Il ruolo del team di sviluppo viene ricoperto dagli sviluppatori, i quali vengono divisi in gruppi e assegnati ai progetti in fase di sviluppo.
\end{itemize}

\begin{figure}[H]
  \centering
  \includesvg[width=\textwidth]{scrum} 
  \caption{Framework Scrum}
  \label{fig:scrum}
\end{figure}

\subsubsection{Issue Tracking System}
L'azienda utilizza Plane\footcite{site:plane} come sistema di tracciamento delle attività. 
Le \gls{issueg} sono organizzate su una board in stile Kanban composta da cinque colonne:
\begin{itemize}
    \item \textit{Backlog}: raccolta iniziale delle attività proposte, non ancora pianificate;
    \item \textit{Todo}: attività prioritarie selezionate per l'esecuzione;
    \item \textit{In Progress}: attività in lavorazione;
    \item \textit{In Review}: attività concluse e in fase di verifica;
    \item \textit{Done}: attività completate.
\end{itemize}
\newpage
