\chapter{Il contesto aziendale}
\label{chap:contesto-aziendale}
\section{Metodologia aziendale}
Spazio Dev è un'azienda di piccole dimensioni, di recente fondazione e ancora in fase di crescita, di conseguenza il progetto è stato sviluppato con budget limitato e da una singola persona.
I processi di sviluppo aziendali, pertanto, sono ancora in rapida evoluzione.
\subsection{Sviluppo software}
\subsubsection{Version Control System}
L'azienda utilizza Git\footcite{site:git} come \gls{VersionControlSystemg} e Gitea\footcite{site:gitea} come piattaforma di hosting per i repository.
Per quanto riguarda il \gls{GitBranchingModelg} viene adottato il Gitflow workflow\footcite{site:gitflow-workflow}, adattato nel seguente modo:
\begin{itemize}
    \item il branch dedicato alle versioni stabili è denominato \textit{main};
    \item il branch dedicato alle versioni di sviluppo è denominato \textit{develop};
    \item i branch dedicati alle implementazioni di \gls{featureg} vengono denominati con lo schema \textit{feature/<nome\_feature>} dove il parametro \textit{<nome\_feature>} è una breve descrizione della \textit{feature} che si sta codificando;
    \item i branch dedicati ai \gls{bugfixg} vengono denominati con lo schema \textit{fix/<nome\_bug>} dove il parametro \textit{<nome\_bug>} è una breve descrizione del \gls{bugg} che si sta risolvendo.
\end{itemize}
Viene inoltre adottato lo standard Conventional Commits 1.0.0\footcite{site:conventional-commits} per la scrittura dei messaggi di commit, in particolare
ogni commit possiede un messaggio così formato: \textit{<tipo>: <descrizione>}, dove il parametro
\textit{<tipo>} è:
\begin{itemize}
    \item \textit{fix} in caso di commit contenente \textit{bugfix};
    \item \textit{feat} in caso di commit contenente \textit{feature}.
\end{itemize}

\subsubsection{Modello di lavoro agile}
\subsubsection{Issue Tracking System}
\subsection{Gestione della coda di stampa}
\section{Cenni sul funzionamento della stampa additiva}
\subsection{}
\newpage