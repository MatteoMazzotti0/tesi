\chapter{Descrizione dello stage}
\label{chap:descrizione-stage}

\section{Introduzione al progetto}
Come descritto nell'introduzione (\ref{chap:introduzione}) il progetto codificato durante il periodo di stage curricolare è uno schedulatore automatico
che ottimizza e riordina la coda di stampa a partire dagli ordini di produzione presenti all'interno del software gestionale Idrotech Manager.
Tale schedulatore deve tenere conto delle ottimizzazioni descritte nella sezione dedicata (\ref{section:printing-management}), oltre a doversi integrare
con il gestionale in utilizzo dal personale.

\section{Analisi preventiva dei rischi}
\label{section:risk-analysis}
La prima fase dello stage curricolare è stata dedicata all'analisi preventiva dei rischi.
Tale procedura ha lo scopo di delineare i possibili rischi a cui si può andare incontro durante lo sviluppo del progetto, oltre a 
trovare delle possibili soluzioni per mitigare tali problematiche.

\begin{risk}{Errata scelta delle tecnologie}
    \riskdescription{una scelta errata delle tecnologie da utilizzare per la codifica del progetto potrebbe portare ad un risultato inutilizzabile o non abbastanza performante}
    \risksolution{coinvolgimento del responsabile nella scelta e occupare il primo periodo per validare le idee proposte}
    \label{risk:tech-stack-choice} 
\end{risk}

\begin{risk}{Sicurezza nell'integrazione del modulo}
    \riskdescription{la comunicazione tra il modulo da sviluppare e il gestionale esistente deve rispettare degli standard minimi di sicurezza, in modo che non tutti possano accedere alle \textit{API} esposte}
    \risksolution{dedicare parte del tempo allo studio su come rendere più sicuro l'accesso alle rotte \textit{API}}
    \label{risk:security} 
\end{risk}

\begin{risk}{Risultati di ordinamento insoddisfacenti}
    \riskdescription{il modulo potrebbe restituire risultati non soddisfacenti, con conseguenti inefficienze nella produzione}
    \risksolution{validare assieme al tutor interno le logiche implementate in maniera periodica, testare il modulo con dataset realistici}
    \label{risk:functioning} 
\end{risk}

\begin{risk}{Prestazioni dello schedulatore}
    \riskdescription{il modulo potrebbe impiegare troppo tempo per ottimizzare la coda di stampa}
    \risksolution{utilizzare un \gls{deployServerg} con buone prestazioni, velocizzare l'esecuzione diminuendo, ad esempio, il tempo disponibile per la ricerca di una soluzione ottima}
    \label{risk:performance} 
\end{risk}

\begin{risk}{Rispetto delle scadenze}
    \riskdescription{le tempistiche dello stage curricolare potrebbero non essere sufficienti per avere un risultato concreto e utilizzabile}
    \risksolution{svolgere una pianificazione (\ref{section:stage-schedule}) accurata del periodo di tempo a disposizione e convalidare il raggiungimento degli obiettivi}
    \label{risk:time} 
\end{risk}

\section{Pianificazione}
\label{section:stage-schedule}
Oltre all'analisi preventiva dei rischi (\ref{section:risk-analysis}) è stata svolta una pianificazione accurata di tutte le fasi del periodo di stage.
Questo, oltre a mitigare il rischio "Rispetto delle scadenze" (\ref{risk:time}) serve a determinare i contenuti da revisionare al termine di ogni \textit{sprint}.

\begin{center}
    \rowcolors{1}{}{tableGray}
    \begin{longtable}{|p{2.25cm}|p{8cm}|p{1cm}|}
    \hline
    \multicolumn{1}{|c|}{\textbf{Settimana}} & \multicolumn{1}{c|}{\textbf{Obiettivi}} & \multicolumn{1}{c|}{\textbf{Ore}}\\ 
    \hline 
    \endfirsthead
    \rowcolor{white}
    \multicolumn{2}{c}{{\bfseries \tablename\ \thetable{} -- Continuo della tabella}}\\
    \hline
    \multicolumn{1}{|c|}{\textbf{A}} & \multicolumn{1}{c|}{B}\\ \hline 
    \endhead
    \hline
    \rowcolor{white}
    \multicolumn{2}{|r|}{{Continua nella prossima pagina...}}\\
    \hline
    \endfoot
    \endlastfoot 
    
        1 & Analisi dei requisiti e dei rischi del progetto, scelta e inizio studio delle tecnologie. & 40 \\
    \hline
        2 & Lettura della documentazione, studio del gestionale e progettazione delle integrazioni. & 40 \\
    \hline
        3 & Inizio della codifica del progetto. & 40  \\
    \hline
        4 & Codifica del progetto. & 40  \\
    \hline
        5 & Codifica del progetto. & 40  \\
    \hline
        6 & Codifica del progetto. & 40  \\
    \hline
        7 & Codifica del progetto, validazione logiche implementate. & 40  \\
    \hline
        8 & Test dell'algoritmo con dataset realistici. & 20  \\
    \hline

    \hiderowcolors
    \caption{Tabella riassuntiva della pianificazione del periodo di stage.}
    \label{tab:pianificazione}
    \end{longtable}
\end{center}


\newpage