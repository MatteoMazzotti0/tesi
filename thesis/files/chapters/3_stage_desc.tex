\chapter{Descrizione dello stage}
\label{chap:descrizione-stage}

\section{Introduzione al progetto}
Come descritto nella \nameref{chap:introduzione} il progetto codificato durante il periodo di stage curricolare è uno schedulatore automatico
che ottimizza e riordina la coda di stampa a partire dagli ordini di produzione presenti all'interno del software gestionale Idrotech Manager.
Tale schedulatore deve tenere conto delle ottimizzazioni descritte nella sezione \nameref{section:printing-management}, oltre a doversi integrare
con il gestionale in utilizzo dal personale.

\section{Analisi preventiva dei rischi}

La prima fase dello stage curricolare è stata dedicata all'analisi preventiva dei rischi.
Tale procedura ha lo scopo di delineare i possibili rischi a cui si può andare incontro durante lo sviluppo del progetto, oltre a 
trovare delle possibili soluzioni per mitigare tali problematiche.

\begin{risk}{Errata scelta delle tecnologie}
    \riskdescription{una scelta errata delle tecnologie da utilizzare per la codifica del progetto potrebbe portare ad un risultato inutilizzabile o non abbastanza performante}
    \risksolution{coinvolgimento del responsabile nella scelta e occupare il primo periodo per validare le idee proposte}
    \label{risk:tech-stack-choice} 
\end{risk}

\begin{risk}{Errata scelta delle tecnologie}
    \riskdescription{una scelta errata delle tecnologie da utilizzare per la codifica del progetto potrebbe portare ad un risultato inutilizzabile o non abbastanza performante}
    \risksolution{coinvolgimento del responsabile nella scelta e occupare il primo periodo per validare le idee proposte}
    \label{risk:tech-stack-choice} 
\end{risk}

\section{Requisiti e obiettivi}

\begin{center}
    \rowcolors{1}{}{tableGray}
    \begin{longtable}{|p{2.25cm}|p{7.75cm}|p{2.25cm}|}
    \hline
    \multicolumn{1}{|c|}{\textbf{A}} & \multicolumn{1}{c|}{\textbf{B}}\\ 
    \hline 
    \endfirsthead
    \rowcolor{white}
    \multicolumn{3}{c}{{\bfseries \tablename\ \thetable{} -- Continuo della tabella}}\\
    \hline
    \multicolumn{1}{|c|}{\textbf{A}} & \multicolumn{1}{c|}{B}\\ \hline 
    \endhead
    \hline
    \rowcolor{white}
    \multicolumn{3}{|r|}{{Continua nella prossima pagina...}}\\
    \hline
    \endfoot
    \endlastfoot 
    
    AA & BB \\
    \hline
    AA & BB \\
    \hline
    AA & BB \\
    \hline
    AA & BB \\
    \hline
    \hiderowcolors
    \caption{Lorem.}
    \label{tab:requisiti_obbiettivi}
    \end{longtable}
\end{center}

\section{Pianificazione}
\begin{figure}[H]
    \centering
    \includegraphics[alt={Testo alternativo dell'immagine}, width=0.5\columnwidth]{img/pk_estate.jpeg}
    \caption{Caption}
    \label{fig:pk_estate_2}
\end{figure}
\lipsum[1]

\subsection{subsection}
\lipsum[1]

\subsubsection{subsubsection}
\lipsum[1]

\paragraph{paragraph}
\lipsum[1]

\newpage