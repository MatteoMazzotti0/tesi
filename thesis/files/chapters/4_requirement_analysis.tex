\chapter{Analisi dei requisiti}
\label{chap:analisi-requisiti}
\section{Casi d'uso}
Come descritto nella sezione \nameref{section:stage-schedule} la prima parte del periodo di stage è stata dedicata all'analisi dei requisiti del progetto.
Prima di fare ciò sono stati definiti tutti i \gls{useCasesg} del sistema.
\subsection{Attori individuati}
L'utente finale del sistema non utilizzerà direttamente il modulo di pianificazione, le sue funzionalità potranno essere messe a disposizione attraverso
il software gestionale, per questo sono stati individuati due principali attori:
\begin{itemize}
    \item operatore, ovvero l'utente che interagisce con il software gestionale;
    \item software gestionale, il quale interagisce con il modulo di pianificazione.
\end{itemize}

Trattandosi di un sistema con il compito di automatizzare un processo, il numero dei \textit{casi d'uso} risulta limitato.
\subsection{Tabella dei casi d'uso}
\tablefirsthead{\hline \textbf{Identificativo} & \textbf{Descrizione} \\ \hline}
\tablehead{\hline \multicolumn{2}{r}{\small Continua...} \\ \hline
           \textbf{Identificativo} & \textbf{Descrizione} \\ \hline}
\tabletail{\hline \multicolumn{2}{r}{\small Continua...} \\ \hline}
\tablelasttail{\hline}

\tablefirsthead{
  \hiderowcolors
  \hline
  \textbf{Identificativo} & \textbf{Descrizione} \\
  \hline
  \showrowcolors
}
\tablehead{
  \hiderowcolors
  \hline \multicolumn{2}{r}{\small Continua...} \\ \hline
  \textbf{Identificativo} & \textbf{Descrizione} \\
  \hline
  \showrowcolors
}
\tabletail{
  \hline \multicolumn{2}{r}{\small Continua...} \\ \hline
}
\tablelasttail{\hline}

{
\centering
\rowcolors{2}{tableGray}{}

\begin{supertabular}{|p{2.75cm}|p{8cm}|}
\hline
UC1 & \textit{Attori}: Gestionale.\newline
      \textit{Scopo}: Ricevere una lista di ordini da pianificare dal software gestionale.\newline
      \textit{Pre-condizione}: Endpoint raggiungibile, autenticazione valida e payload validato correttamente.\newline
      \textit{Post-condizione}: Richiesta accettata e messa in coda per l'elaborazione. \\ \hline
UC1.1 & \textit{Attori}: Gestionale.\newline
      \textit{Scopo}: Calcolare un piano di stampa coerente con i vincoli.\newline
      \textit{Pre-condizione}: Esiste una richiesta di schedulazione accettata.\newline
      \textit{Post-condizione}: Piano di schedulazione prodotto e pronto all'invio. \\ \hline
UC1.2 & \textit{Attori}: Gestionale.\newline
      \textit{Scopo}: Notificare al gestionale l'esito della schedulazione. \newline
      \textit{Pre-condizione}: Risultato disponibile, webhook del gestionale configurato e raggiungibile.\newline
      \textit{Post-condizione}: Esito notificato e inviato al gestionale. \\ \hline
UC1.E & \textit{Attori}: Gestionale.\newline
      \textit{Scopo}: Comunicare un errore occorso in UC1 o UC1.1.\newline
      \textit{Pre-condizione}: Si è verificata un'anomalia (mancata autorizzazione, validazione dei dati fallita o altro).\newline
      \textit{Post-condizione}: Errore tracciato e inviato al gestionale. \\ \hline
UC2 & \textit{Attori}: Operatore.\newline
     \textit{Scopo}: Consultare l'elenco dei risultati di schedulazione.\newline
     \textit{Pre-condizione}: Schedulazioni precedenti presenti.\newline
     \textit{Post-condizione}: Schedulazioni mostrate. \\ \hline
UC2.1 & \textit{Attori}: Operatore.\newline
       \textit{Scopo}: Visualizzare il piano su vista a calendario.\newline
       \textit{Pre-condizione}: Almeno una schedulazione presente nel database.\newline
       \textit{Post-condizione}: Calendario visualizzabile a schermo con gli slot pianificati. \\ \hline
UC2.2 & \textit{Attori}: Operatore.\newline
     \textit{Scopo}: Confermare l'accettazione della coda proposta.\newline
     \textit{Pre-condizione}: Sezione dei dettagli della schedulazione aperta.\newline
     \textit{Post-condizione}: Schedulazione marcata come accettata e importazione avvenuta nel calendario. \\ \hline
UC2.E & \textit{Attori}: Operatore.\newline
     \textit{Scopo}: Informare l'utente di errori avvenuti durante l'importazione a calendario.\newline
     \textit{Pre-condizione}: Errore occorso in UC2.2.\newline
     \textit{Post-condizione}: Errore comunicato e stato invariato. \\ \hline
UC3 & \textit{Attori}: Operatore.\newline
      \textit{Scopo}: Richiedere l'avvio della schedulazione automatica.\newline
      \textit{Pre-condizione}: Operatore autenticato, sezione Ordini accessibile.\newline
      \textit{Post-condizione}: Richiesta inviata al modulo e accettata. \\ \hline
UC3.1 & \textit{Attori}: Operatore.\newline
       \textit{Scopo}: Selezionare uno o più ordini da includere nella schedulazione.\newline
       \textit{Pre-condizione}: Modale di selezione ordini aperto.\newline
       \textit{Post-condizione}: Ordini da pianificare selezionati e pronti all'invio. \\ \hline
UC3.2 & \textit{Attori}: Operatore.\newline
       \textit{Scopo}: Informare dell'avvenuta accettazione della richiesta da parte del modulo.\newline
       \textit{Pre-condizione}: Risposta positiva del modulo.\newline
       \textit{Post-condizione}: Notifica visualizzata, modale chiuso. \\ \hline
UC3.E & \textit{Attori}: Operatore\newline
     \textit{Scopo}: Informare l'utente di errori incontrati durante l'avvio della schedulazione.\newline
     \textit{Pre-condizione}: Errore occorso in UC3.\newline
     \textit{Post-condizione}: Errore comunicato e stato invariato. \\ \hline

\end{supertabular}
\par
}

\subsection{Diagrammi dei casi d'uso}
I diagrammi dei casi d'uso sono riportati nelle figure \ref{fig:uc-first}, \ref{fig:uc-second} e \ref{fig:uc-third}.
\begin{figure}[H]
  \centering
  \includegraphics[alt={Diagramma dei casi d'uso UC1, UC1.1, UC1.2, UC1.E}, width=\textwidth]{img/usecase/1.png}
  \caption{Diagramma dei casi d'uso UC1, UC1.1, UC1.2, UC1.E}
  \label{fig:uc-first}
\end{figure}



\begin{figure}[H]
  \centering
  \includegraphics[alt={Diagramma dei casi d'uso UC2, UC2.1, UC2.2, UC2.E}, width=\textwidth]{img/usecase/2.png}
  \caption{Diagramma dei casi d'uso UC2, UC2.1, UC2.2, UC2.E}
  \label{fig:uc-second}
\end{figure}

\begin{figure}[H]
  \centering
  \includegraphics[alt={Diagramma dei casi d'uso UC3, UC3.1, UC3.2, UC3.E}, width=\textwidth]{img/usecase/4.png}
  \caption{Diagramma dei casi d'uso UC3, UC3.1, UC3.2, UC3.E}
  \label{fig:uc-third}
\end{figure}

\section{Analisi dei requisiti}
Una volta stilata la lista dei \textit{casi d'uso} è stato possibile procedere con l'effettiva analisi dei requisiti del sistema.
Tale operazione si è svolta attraverso un incontro con il tutor aziendale, durante il quale sono stati rilevati tutti i requisiti obbligatori, desiderabili e facoltativi.
\subsection{Tracciamento dei requisiti}
Ogni requisito viene identificato attraverso un codice identificativo di questo tipo:
\begin{center}
R.x.y
\end{center}
dove:
\begin{itemize}
  \item la lettera R identifica il requisito;
  \item la lettera x identifica la tipologia di tale requisito, ovvero:
 \begin{itemize}
    \item O obbligatorio;
    \item D desiderabile;
    \item F facoltativo.
  \end{itemize}
  \item la lettera y è un valore numerico progressivo a partire da 0.
\end{itemize}
\newpage
\section{Tabelle dei requisiti}
\subsection{Tabella dei requisiti obbligatori}
\begin{table}[htbp]
    \centering
    \rowcolors{1}{}{tableGray}
    \begin{tabular}{|p{2.25cm}|p{10cm}|}
    \hline
    \multicolumn{1}{|c|}{\textbf{Requisito}} & \multicolumn{1}{c|}{\textbf{Descrizione}} \\
    \hline
    RO0 & Il modulo deve produrre una coda di stampa dettagliata, evidenziando quale articolo stampare e quale stampante usare.\\
    \hline 
    RO1 & La schedulazione proposta deve considerare gli orari di lavoro aziendali. \\
    \hline 
    RO2 & La schedulazione proposta deve ottimizzare la stampa raggruppando i lavori con lo stesso materiale.  \\
    \hline 
    RO3 & La schedulazione proposta deve ottimizzare la stampa pianificando i lavori corti durante la giornata lavorativa.  \\
    \hline 
    RO4 & La schedulazione proposta deve ottimizzare la stampa pianificando i lavori lunghi al di fuori della giornata lavorativa.  \\
    \hline 
    RO5 & La schedulazione deve essere avviabile dal software gestionale esistente.  \\
    \hline 
    RO6 & La schedulazione deve essere visualizzabile dal software gestionale.  \\
    \hline 
    RO7 & La schedulazione deve essere importabile nel calendario del software gestionale.  \\
    \hline 
    RO8 & La schedulazione deve essere visibile in maniera chiara, in modo che l'utente possa trovare facilmente le informazioni utili.  \\
    \hline 
    RO9 & La soluzione deve essere ammissibile e non devono esserci lavori sovrapposti.\\
    \hline 
    RO10 & Tutti i pezzi da stampare devono essere inseriti nella pianificazione, non sono ammesse stampe parziali. \\
    \hline 
    RO11 & Deve essere considerato un tempo di cambio piatto tra una stampa e l'altra. \\
    \hline 
    RO12 & Deve essere considerato un tempo di cambio filamento tra un lavoro di stampa e il suo successivo se usano materiali diversi. \\
    \hline
    \end{tabular}
    \caption{Tabella del tracciamento dei requisiti obbligatori.}
    \label{tab:mandatory-requirements}
\end{table}
\newpage
\subsection{Tabella dei requisiti desiderabili}
\begin{table}[htbp]
    \centering
    \rowcolors{1}{}{tableGray}
    \begin{tabular}{|p{2.25cm}|p{10cm}|}
    \hline
    \multicolumn{1}{|c|}{\textbf{Requisito}} & \multicolumn{1}{c|}{\textbf{Descrizione}} \\
    \hline
    RD0 & Deve essere disponibile un metodo di autenticazione per validare l'origine della richiesta delle schedulazioni.  \\
    \hline
    RD1 & Lo sviluppo deve avvenire utilizzando \gls{designPatterng} comprovati e producendo codice mantenibile per facilitare le modifiche future.\\
    \hline
    RD2 & La schedulazione proposta deve ottimizzare la stampa raggruppando i lavori con lo stesso ugello.\\
    \hline
    RD3 & La schedulazione proposta deve ottimizzare la stampa raggruppando i lavori con lo stesso colore.\\
    \hline
    RD4 & Al termine dello sviluppo l'algoritmo implementato deve essere validato utilizzando set di dati realistici.\\
    \hline
    RD5 & La schedulazione proposta deve ottimizzare il numero di pezzi stampati nel piatto, minimizzando gli sprechi e il ritardo nelle consegne.\\
    \hline
    \end{tabular}
    \caption{Tabella del tracciamento dei requisiti desiderabili.}
    \label{tab:desirable-requirements}
\end{table}
\subsection{Tabella dei requisiti facoltativi}
\begin{table}[htbp]
    \centering
    \rowcolors{1}{}{tableGray}
    \begin{tabular}{|p{2.25cm}|p{10cm}|}
    \hline
    \multicolumn{1}{|c|}{\textbf{Requisito}} & \multicolumn{1}{c|}{\textbf{Descrizione}} \\
    \hline
    RF0 & La pianificazione deve avvenire automaticamente ad intervalli di tempo prefissati e modificabili dall'utente.  \\
    \hline
    RF1 & La pianificazione deve poter essere calcolata a partire da uno stato iniziale, schedulando solo i lavori mancanti.  \\
    \hline
    RF2 & L'utente deve essere in grado di modificare i parametri dello schedulatore dal software gestionale.  \\
    \hline
    \end{tabular}
    \caption{Tabella del tracciamento dei requisiti facoltativi.}
    \label{tab:optional-requirements}
\end{table}

\newpage
