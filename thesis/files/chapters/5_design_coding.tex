\chapter{Progettazione e codifica}
\label{chap:progettazione-codifica}
La fase di progettazione e ccodifica è stata la più significativa per il progetto, in quanto ha occupato la maggior parte del tempo a disposizione.
Attraverso un incontro con il tutor aziendale sono stati definiti i \textit{framework} e le librerie da utilizzare per la codifica,
inoltre sono state prese decisioni in merito all'architettura del sistema e ai \textit{design pattern} da utilizzare.
\section{Tecnologie e strumenti}
\label{sec:tecnologie-strumenti}
Di seguito viene data una panoramica delle tecnologie e strumenti utilizzati.
\subsection{Linguaggi e framework}
\subsubsection{Python}
Linguaggio di programmazione ad alto livello, orientato agli oggetti e adatto a sviluppare applicazioni distribuite.
Viene utilizzato nello sviluppo web, nell'analisi dei dati, nel machine learning e nello sviluppo di automazioni.
\subsubsection{FastAPI}
\Gls{frameworkg} web per la codifica di \textit{API} basato su Python\footcite{site:python}.
Offre elevate prestazioni, validazione dell'input attraverso l'utilizzo della libreria Pydantic\footcite{site:pydantic} e 
documentazione automatica seguendo lo standard OpenAPI\footcite{site:openapi}.
\subsection{Librerie}
\subsubsection{Black}
Libreria che permette di formattare il codice Python, utilizzata per imporre un \gls{codestyleg} unico e automatizzabile.
\subsubsection{isort}
Libreria che permette di formattare le importazioni delle librerie Python, utilizzata per imporre un \textit{codestyle} unico e automatizzabile.
\subsubsection{SQLModel}
Libreria che permette di interagire con basi di dati relazionali attraverso il codice scritto in Python con 
la tecnica di programmazione \gls{orm}.
\subsubsection{PyJWT}
Libreria che permette di codificare e decodificare \gls{jwt}, utilizzati per autenticare utenti o servizi.
\subsubsection{Pydantic}
Libreria che aggiunge funzionalità legate alla validazione dei dati in Python, permette di definire con precisione la struttura dei dati
e gestire i casi in cui si riceva un input non conforme alle aspettative.

\subsubsection{Google OR-Tools}
Libreria che contiene strumenti per risolvere problemi di \gls{combinatorialOptimizationg}.
Permette di modellare un problema attraverso diversi linguaggi di programmazione (C++, Python, C\# o Java) e utilizzare
diversi risolutori commerciali o \gls{openSourceg}.
\subsection{DevOps e server di deploy}
\subsubsection{Docker}
Piattaforma utilizzata per creare ambienti isolati e riproducibili dove eseguire le applicazioni. 
Consente di standardizzare lo sviluppo locale e il \gls{deployg}, riducendo differenze tra ambienti e semplificando l'esecuzione dei servizi.
\subsubsection{Uvicorn}
Server di tipo \gls{asgi} per applicazioni web Python asincrone. Utilizzato per eseguire l'\textit{API} in sviluppo e in produzione, supporta la concorrenza.
\subsection{Persistenza dei dati}
\subsubsection{MySQL}
Sistema di gestione di basi di dati relazionali usato per la persistenza. 
Supporta transazioni e indici, ed è integrato con l'applicazione tramite \textit{orm} per la definizione dello schema e l'accesso ai dati.

\section{Scelte implementative}
\subsection{Progettazione della base di dati}
\subsubsection{Base di dati del modulo di pianificazione}
Assieme al tutor aziendale è stata presa la decisione di non utilizzare la base di dati del modulo di pianificazione per la persistenza dei risultati generati,
questo perchè la coda di stampa viene visualizzata direttamente all'interno del software gestionale e risulta pertanto più semplice e intuitivo 
modificare direttamente il \textit{database} di quest'ultimo.
La base di dati del modulo di pianificazione ha il solo compito di memorizzare i \gls{clientg} registrati e autorizzati a chiamare le
\textit{API} esposte, a tale scopo è stata creata una sola tabella denominata \textit{clients} e così strutturata:
\begin{itemize}
    \item \textit{id}: intero auto-incrementale utilizzato come chiave primaria, generato automaticamente dal \textit{database}.
    \item \textit{client\_id}: stringa obbligatoria indicizzata e univoca che identifica il client registrato.
    \item \textit{client\_secret}: stringa obbligatoria che contiene la password utilizzata dal client per l'autenticazione (vedi \ref{section:system-security}).
    \item \textit{created\_at}: campo di tipo \textit{DateTime} per tracciare la data di registrazione.
    \item \textit{is\_active}: valore booleano che indica se il client è abilitato, è valorizzato a \textit{true} di default.
\end{itemize}
Ciò permette, in futuro, di aggiungere eventuali altri \textit{client} autorizzati ad utilizzare il modulo.
\subsubsection{Base di dati del software gestionale}
La progettazione della base di dati è risultata fondamentale per permettere all'utente di visualizzare correttamente le code di stampa pianificate.
Il software gestionale esistente (e già sviluppato al momento della codifica del progetto) possiede una base di dati già strutturata e utilizzata sia dall'azienda ospitante che da
alcuni clienti dell'azienda stessa (in quanto il software è \gls{multitenantg}).
Per questo motivo l'ampliamento della base di dati esistente è stato eseguito cercando di mantenere al minimo il numero di modifiche.


\subsection{Architettura del sistema}
Il \gls{architecturalDesignPatterng} scelto per la codifica del progetto è il \gls{layerPatterng}.
Tale scelta è stata presa per le seguenti motivazioni:
\begin{itemize}
    \item Permette una buona separazione delle responsabilità, ogni \textit{layer} non conosce infatti l'implementazione degli altri;
    \item Permette una maggior semplicità di sviluppo, in quanto è un \textit{design pattern architetturale} conosciuto e di semplice comprensione;
    \item Permette una buona manutenibilità del codice, ogni \textit{layer} non viene influenzato dai cambiamenti apportati ad altri layer (a patto che i \textit{contratti} tra i layer rimangano invariati);
    \item Permette una buona testabilità, ogni \textit{layer} può essere isolato e testato separatamente;
    \item Il progetto non richiede una elevata \gls{scalabilityg} del sistema.
\end{itemize}
\subsection{Design pattern adottati}
\subsection{Struttura delle cartelle}
\section{Sicurezza del sistema}
\label{section:system-security}
\newpage
