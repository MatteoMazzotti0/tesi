\chapter{Conclusioni}
\label{chap:conclusioni}
\section{Copertura dei requisiti}
La tabella \ref{tab:requirements-coverage} riporta la copertura dei requisiti, ovvero un resoconto completo dei requisiti soddisfatti e non soddisfatti.
\begin{table}[htbp]
    \centering
    \rowcolors{1}{}{tableGray}
    \begin{tabular}{|p{2.75cm}|p{2.75cm}|}
    \hline
    \multicolumn{1}{|c|}{\textbf{Identificativo}} & \multicolumn{1}{c|}{\textbf{Soddisfatto}} \\
    \hline
    RO0 & Sì\\
    \hline 
    RO1 & Sì \\
    \hline 
    RO2 & Sì  \\
    \hline 
    RO3 & Sì  \\
    \hline 
    RO4 & Sì  \\
    \hline 
    RO5 & Sì  \\
    \hline 
    RO6 & Sì  \\
    \hline 
    RO7 & Sì  \\
    \hline 
    RO8 & Sì  \\
    \hline 
    RO9 & Sì   \\
    \hline 
    RO10 & Sì \\
    \hline 
    RO11 & Sì \\
    \hline 
    RO12 & Sì \\
    \hline
    RD0 & Sì  \\
    \hline
    RD1 & Sì  \\
    \hline
    RD2 & No   \\
    \hline
    RD3 & No   \\
    \hline
    RD4 & Sì   \\
    \hline
    RD5 & Sì   \\
    \hline
    RF0 & No   \\
    \hline
    RF1 & No  \\
    \hline
    RF2 & No  \\
    \hline
    \end{tabular}
    \caption{Tabella del tracciamento del soddisfacimento dei requisiti.}
    \label{tab:requirements-coverage}
\end{table}
\section{Analisi del prodotto}
In questa sezione viene presentata un'analisi critica del prodotto consegnato all'azienda ospitante.
\subsection{Prodotto}
Il prodotto sviluppato rappresenta sicuramente un buon contributo all'automazione della gestione degli ordini di produzione.
Il sistema è stato integrato correttamente con il software gestionale esistente ed è quindi immediatamente utilizzabile dall'azienda.
È possibile inoltre continuare a estendere e a migliorare il codice, in quanto sono stati adottati \textit{design pattern} specifici per aumentare 
la manutenibilità.
Uno dei punti critici dello schedulatore è sicuramente rappresentato dalle sue prestazioni, in quanto il sistema impiega molto tempo per il calcolo di una soluzione ottima.
L'implementazione del \textit{batching} riesce a mitigare il problema, a patto che una soluzione sub-ottima sia sufficiente per l'utente.
Non è stato inoltre possibile, per mancanza di tempo, valutare la differenza media tra una soluzione ottima e una soluzione sub-ottima.
\subsubsection{Miglioramenti}
Il prodotto consegnato può essere migliorato nei seguenti modi:
\begin{itemize}
    \item Implementazione dei requisiti desiderabili e facoltativi mancanti;
    \item Miglioramento delle performance tramite ottimizzazione del numero di variabili generate.
\end{itemize}
\newpage