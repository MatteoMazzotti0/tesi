% Acronyms
\newacronym{api}{API}{Application Program Interface}
\newacronym{sdk}{SDK}{Software Development Kit}
\newacronym{uml}{UML}{Unified Modeling Language}
\newacronym{tsa}{TSA}{Termine solo acronimo}
\newacronym{poc}{POC}{Proof Of Concept}
\newacronym{orm}{ORM}{Object-Relational Mapping}
\newacronym{jwt}{JWT}{JSON Web Token}
\newacronym{asgi}{ASGI}{Asynchronous Server Gateway Interface}

% Glossary
\newglossaryentry{apig}{
    name={API},
    text={Application Program Interface},
    sort=api,
    description={In informatics, an API is a set of procedures available to programmers, typically grouped to form a toolkit for a specific task within a program. Its purpose is to provide an abstraction, usually between hardware and the programmer or between low-level and high-level software, simplifying the programming process}
}

\newglossaryentry{sdkg}{
    name={SDK},
    text={Software Development Kit},
    sort=sdk,
    description={A Software Development Kit (SDK) is a collection of development tools in one installable package, facilitating application creation by providing a compiler, debugger, and sometimes a software framework. SDKs are typically specific to a hardware platform and operating system combination. Many application developers use specific SDKs to enable advanced functionalities such as advertisements, push notifications, etc}
}

\newglossaryentry{umlg}{
    name={UML},
    text={Unified Modeling Language},
    sort=uml,
    description={In software engineering, Unified Modeling Language (UML) is a modeling and specification language based on the object-oriented paradigm. UML serves as a "lingua franca" in the object-oriented design and programming community. Much of the industry literature uses UML to describe analytical and design solutions in a concise and understandable way for a broad audience}
}

\newglossaryentry{TermineSenzaAcronimo}{
    name={Nome del termine},
    sort=termine senza acronimo,
    description={Descrizione}
}

\newglossaryentry{pocg}{
    name={POC},
    text={Proof Of Concept},
    sort=poc,
    description={Il Proof Of Concept (POC) è l'allestimento di una demo prototipale del sistema o applicazione in sviluppo o in corso di valutazione}
}

\newglossaryentry{VersionControlSystemg}{
    name={Version Control System},
    text={Version Control System},
    sort=Version Control System,
    description={Un sistema di versionamento (Version Control System o VCS) è uno strumento software che traccia e gestisce le modifiche apportate a un file o a un insieme di file nel tempo, permettendo di recuperare versioni precedenti e di collaborare con altri utenti}
}

\newglossaryentry{GitBranchingModelg}{
    name={Git branching model},
    text={Git branching model},
    sort=Git branching model,
    description={Un modello di branching Git è una strategia o un insieme di regole che definisce come i team devono creare, gestire e unire i branch in un repository Git, al fine di organizzare il flusso di sviluppo}
}

\newglossaryentry{featureg}{
    name={Feature},
    text={feature},
    sort=feature,
    description={Unità coerente di comportamento di un sistema che produce un beneficio osservabile per l'utente}
}

\newglossaryentry{bugg}{
    name={Bug},
    text={bug},
    sort=bug,
    description={In informatica, errore di funzionamento di un sistema o di un programma}
}


\newglossaryentry{bugfixg}{
    name={Bugfix},
    text={bugfix},
    sort=bug,
    description={modifica del codice, della configurazione o dei dati volta a rimuovere un malfunzionamento (bug) e a ripristinare il comportamento atteso del sistema senza introdurre cambiamenti funzionali non richiesti}
}

\newglossaryentry{productOwnerg}{
    name={Product Owner},
    text={Product Owner},
    sort={Product Owner},
    description={Figura chiave di Scrum responsabile di massimizzare il valore del prodotto e del lavoro del team. Definisce e mantiene il Product Backlog, ne ordina gli elementi in base al valore e agli obiettivi, chiarisce i requisiti e accetta l'incremento completato}
}

\newglossaryentry{scrumMasterg}{
    name={Scrum Master},
    text={Scrum Master},
    sort={Scrum Master},
    description={Servant leader del team Scrum. Promuove e supporta Scrum come definito nella Scrum Guide, facilita gli eventi Scrum, rimuove impedimenti, tutela il team e aiuta l'organizzazione ad adottare pratiche agili}
}

\newglossaryentry{sprintPlanningg}{
    name={Sprint Planning},
    text={Sprint Planning},
    sort={Sprint Planning},
    description={Evento che apre lo Sprint in cui lo Scrum Team definisce lo Sprint Goal, seleziona gli elementi del Product Backlog da includere nello Sprint e pianifica il lavoro necessario nel Sprint Backlog}
}

\newglossaryentry{sprintg}{
    name={Sprint},
    text={Sprint},
    sort={Sprint},
    description={Time-box tipico del framework Scrum, della durata compresa fra uno e quattro settimane, durante il quale il team realizza un incremento potenzialmente rilasciabile seguendo uno Sprint Goal condiviso. Lo Sprint include pianificazione, sviluppo, revisione e retrospettiva}
}

\newglossaryentry{dailyScrumg}{
    name={Daily Scrum},
    text={Daily Scrum},
    sort={Daily Scrum},
    description={Evento giornaliero, time-box di 15 minuti, in cui i Developer ispezionano i progressi verso lo Sprint Goal e adattano il piano per le prossime 24 ore; non è una riunione di status per gli stakeholder}
}

\newglossaryentry{sprintReviewg}{
    name={Sprint Review},
    text={Sprint Review},
    sort={Sprint Review},
    description={Evento alla fine dello Sprint per ispezionare l'incremento e adattare il Product Backlog. Coinvolge stakeholder e team per raccogliere feedback, rivedere i risultati e allineare i prossimi passi}
}

\newglossaryentry{sprintRetrospectiveg}{
    name={Sprint Retrospective},
    text={Sprint Retrospective},
    sort={Sprint Retrospective},
    description={Ultimo evento dello Sprint dedicato all'ispezione delle modalità di lavoro del team e all'individuazione di miglioramenti pratici da implementare nel prossimo Sprint}
}

\newglossaryentry{backlogg}{
    name={Backlog},
    text={backlog},
    sort={backlog},
    description={Elenco ordinato e dinamico di elementi di lavoro (item) che rappresentano valore da realizzare. In Scrum, il Product Backlog, di responsabilità del Product Owner, raccoglie e priorizza le esigenze del prodotto; lo Sprint Backlog è il piano dei Developer per raggiungere lo Sprint Goal durante lo Sprint}
}

\newglossaryentry{issueg}{
    name={Issue},
    text={issue},
    sort={issue},
    description={Unità di lavoro o ticket che rappresenta una richiesta, un bug, un miglioramento o un'attività da tracciare e gestire in un sistema di gestione del lavoro come Jira o GitHub Issues. Ogni issue dovrebbe includere contesto, criteri di accettazione e stato per facilitare la collaborazione}
}

\newglossaryentry{designPatterng}{
    name={Design Pattern},
    text={design pattern},
    sort={design pattern},
    description={Soluzione progettuale riutilizzabile che descrive come risolvere un problema ricorrente di design software, codificando ruoli, responsabilità e interazioni tra componenti per migliorare manutenibilità, flessibilità e comunicazione all'interno del team}
}

\newglossaryentry{architecturalDesignPatterng}{
    name={Design pattern architetturale},
    text={design pattern architetturale},
    sort={design pattern architetturale},
    description={Schema di alto livello che definisce composizione e interazione dei componenti di un sistema software per soddisfare requisiti non funzionali come scalabilità, resilienza o sicurezza, fornendo linee guida per strutturare l'intera architettura}
}

\newglossaryentry{layerPatterng}{
    name={Pattern a layer},
    text={pattern a layer},
    sort={pattern a layer},
    description={Design pattern architetturale che suddivide un sistema in strati indipendenti con responsabilità specifiche (es. presentazione, logica, accesso ai dati), così da ridurre le dipendenze e facilitare riuso, manutenzione e test}
}

\newglossaryentry{layerg}{
    name={Layer},
    text={layer},
    sort={layer},
    description={Strato logico dell'architettura che incapsula un insieme coerente di responsabilità e offre servizi attraverso interfacce ben definite agli strati adiacenti, limitando l'accoppiamento e favorendo la separazione delle preoccupazioni}
}

\newglossaryentry{scalabilityg}{
    name={Scalabilità},
    text={scalabilità},
    sort={scalabilita},
    description={Capacità di un sistema di sostenere carichi crescenti (utenti, dati o richieste) mantenendo livelli di servizio accettabili, adattando risorse hardware o struttura software tramite scaling verticale, orizzontale o elastico}
}

\newglossaryentry{deployServerg}{
    name={Server di deploy},
    text={server di deploy},
    sort={server di deploy},
    description={Computer dedicato all'esecuzione in produzione di un'applicazione software. Ospita il codice distribuito, espone le relative API e garantisce risorse hardware e configurazioni adeguate per gestire il carico operativo previsto.}
}

\newglossaryentry{useCasesg}{
    name={Casi d'uso},
    text={casi d'uso},
    sort={casi d'uso},
    description={Descrizioni strutturate di come gli attori interagiscono con il sistema per raggiungere obiettivi specifici, evidenziando flussi principali, varianti ed esigenze funzionali che guidano l'analisi dei requisiti}
}

\newglossaryentry{frameworkg}{
    name={Framework},
    text={framework},
    sort={framework},
    description={Insieme coerente di componenti software riutilizzabili che fornisce struttura, astrazioni e convenzioni per sviluppare una specifica classe di applicazioni, velocizzando lo sviluppo e favorendo la consistenza del codice}
}

\newglossaryentry{codestyleg}{
    name={Code style},
    text={code style},
    sort={code style},
    description={Insieme di regole e convenzioni formali che disciplinano formattazione, nomenclatura e organizzazione del codice sorgente, così da mantenerlo leggibile e uniforme all'interno di un team}
}

\newglossaryentry{ormg}{
    name={Object-Relational Mapping (ORM)},
    text={Object-Relational Mapping (ORM)},
    sort={object-relational mapping},
    description={Tecnica che mappa oggetti e classi di un linguaggio orientato agli oggetti su tabelle e record di un database relazionale, consentendo di manipolare i dati tramite codice anziché query SQL esplicite}
}

\newglossaryentry{jwttg}{
    name={JSON Web Token (JWT)},
    text={JSON Web Token (JWT)},
    sort={json web token},
    description={Standard aperto per la creazione di token JSON firmati e opzionalmente cifrati che rappresentano in modo sicuro affermazioni tra due parti, comunemente usato per autenticazione e scambio di informazioni tra servizi}
}

\newglossaryentry{combinatorialOptimizationg}{
    name={Ottimizzazione combinatoria},
    text={ottimizzazione combinatoria},
    sort={ottimizzazione combinatoria},
    description={Area dell'ottimizzazione che studia problemi in cui si cerca la soluzione migliore tra un insieme finito ma molto ampio di combinazioni discrete, tipicamente soggetti a vincoli (es. scheduling, routing, assegnamento)}
}

\newglossaryentry{openSourceg}{
    name={Open source},
    text={open source},
    sort={open source},
    description={Modello di distribuzione del software in cui il codice sorgente è reso pubblico con una licenza che consente a chiunque di studiarlo, modificarlo e ridistribuirlo, favorendo collaborazione e trasparenza}
}

\newglossaryentry{deployg}{
    name={Deploy},
    text={deploy},
    sort={deploy},
    description={Processo di rilascio e messa in esercizio di un'applicazione su un ambiente target (test, staging, produzione), comprendendo packaging, distribuzione, configurazione e attivazione dei servizi necessari per renderla disponibile agli utenti}
}

\newglossaryentry{asgig}{
    name={Asynchronous Server Gateway Interface (ASGI)},
    text={Asynchronous Server Gateway Interface (ASGI)},
    sort={asynchronous server gateway interface},
    description={Specifiche che definiscono l'interfaccia tra server web e applicazioni Python asincrone, evoluzione non bloccante del WSGI che abilita applicazioni in tempo reale e protocolli multipli (HTTP, WebSocket)}
}
