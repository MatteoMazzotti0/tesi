% Acronyms
\newacronym{api}{API}{Application Program Interface}
\newacronym{sdk}{SDK}{Software Development Kit}
\newacronym{uml}{UML}{Unified Modeling Language}
\newacronym{tsa}{TSA}{Termine solo acronimo}
\newacronym{poc}{POC}{Proof Of Concept}

% Glossary
\newglossaryentry{apig}{
    name={API},
    text={Application Program Interface},
    sort=api,
    description={In informatics, an API is a set of procedures available to programmers, typically grouped to form a toolkit for a specific task within a program. Its purpose is to provide an abstraction, usually between hardware and the programmer or between low-level and high-level software, simplifying the programming process}
}

\newglossaryentry{sdkg}{
    name={SDK},
    text={Software Development Kit},
    sort=sdk,
    description={A Software Development Kit (SDK) is a collection of development tools in one installable package, facilitating application creation by providing a compiler, debugger, and sometimes a software framework. SDKs are typically specific to a hardware platform and operating system combination. Many application developers use specific SDKs to enable advanced functionalities such as advertisements, push notifications, etc}
}

\newglossaryentry{umlg}{
    name={UML},
    text={Unified Modeling Language},
    sort=uml,
    description={In software engineering, Unified Modeling Language (UML) is a modeling and specification language based on the object-oriented paradigm. UML serves as a "lingua franca" in the object-oriented design and programming community. Much of the industry literature uses UML to describe analytical and design solutions in a concise and understandable way for a broad audience}
}

\newglossaryentry{TermineSenzaAcronimo}{
    name={Nome del termine},
    sort=termine senza acronimo,
    description={Descrizione}
}

\newglossaryentry{pocg}{
    name={POC},
    text={Proof Of Concept},
    sort=poc,
    description={Il Proof Of Concept (POC) è l'allestimento di una demo prototipale del sistema o applicazione in sviluppo o in corso di valutazione}
}

\newglossaryentry{VersionControlSystemg}{
    name={Version Control System},
    text={Version Control System},
    sort=Version Control System,
    description={Un sistema di versionamento (Version Control System o VCS) è uno strumento software che traccia e gestisce le modifiche apportate a un file o a un insieme di file nel tempo, permettendo di recuperare versioni precedenti e di collaborare con altri utenti}
}

\newglossaryentry{GitBranchingModelg}{
    name={Git branching model},
    text={Git branching model},
    sort=Git branching model,
    description={Un modello di branching Git è una strategia o un insieme di regole che definisce come i team devono creare, gestire e unire i branch in un repository Git, al fine di organizzare il flusso di sviluppo}
}

\newglossaryentry{featureg}{
    name={Feature},
    text={feature},
    sort=feature,
    description={Unità coerente di comportamento di un sistema che produce un beneficio osservabile per l'utente}
}

\newglossaryentry{bugg}{
    name={Bug},
    text={bug},
    sort=bug,
    description={In informatica, errore di funzionamento di un sistema o di un programma}
}


\newglossaryentry{bugfixg}{
    name={Bugfix},
    text={bugfix},
    sort=bug,
    description={modifica del codice, della configurazione o dei dati volta a rimuovere un malfunzionamento (bug) e a ripristinare il comportamento atteso del sistema senza introdurre cambiamenti funzionali non richiesti}
}

\newglossaryentry{productOwnerg}{
    name={Product Owner},
    text={Product Owner},
    sort={Product Owner},
    description={Figura chiave di Scrum responsabile di massimizzare il valore del prodotto e del lavoro del team. Definisce e mantiene il Product Backlog, ne ordina gli elementi in base al valore e agli obiettivi, chiarisce i requisiti e accetta l'incremento completato}
}

\newglossaryentry{scrumMasterg}{
    name={Scrum Master},
    text={Scrum Master},
    sort={Scrum Master},
    description={Servant leader del team Scrum. Promuove e supporta Scrum come definito nella Scrum Guide, facilita gli eventi Scrum, rimuove impedimenti, tutela il team e aiuta l'organizzazione ad adottare pratiche agili}
}

\newglossaryentry{sprintPlanningg}{
    name={Sprint Planning},
    text={Sprint Planning},
    sort={Sprint Planning},
    description={Evento che apre lo Sprint in cui lo Scrum Team definisce lo Sprint Goal, seleziona gli elementi del Product Backlog da includere nello Sprint e pianifica il lavoro necessario nel Sprint Backlog}
}

\newglossaryentry{dailyScrumg}{
    name={Daily Scrum},
    text={Daily Scrum},
    sort={Daily Scrum},
    description={Evento giornaliero, time-box di 15 minuti, in cui i Developer ispezionano i progressi verso lo Sprint Goal e adattano il piano per le prossime 24 ore; non è una riunione di status per gli stakeholder}
}

\newglossaryentry{sprintReviewg}{
    name={Sprint Review},
    text={Sprint Review},
    sort={Sprint Review},
    description={Evento alla fine dello Sprint per ispezionare l'incremento e adattare il Product Backlog. Coinvolge stakeholder e team per raccogliere feedback, rivedere i risultati e allineare i prossimi passi}
}

\newglossaryentry{sprintRetrospectiveg}{
    name={Sprint Retrospective},
    text={Sprint Retrospective},
    sort={Sprint Retrospective},
    description={Ultimo evento dello Sprint dedicato all'ispezione delle modalità di lavoro del team e all'individuazione di miglioramenti pratici da implementare nel prossimo Sprint}
}

\newglossaryentry{backlogg}{
    name={Backlog},
    text={backlog},
    sort={backlog},
    description={Elenco ordinato e dinamico di elementi di lavoro (item) che rappresentano valore da realizzare. In Scrum, il Product Backlog, di responsabilità del Product Owner, raccoglie e priorizza le esigenze del prodotto; lo Sprint Backlog è il piano dei Developer per raggiungere lo Sprint Goal durante lo Sprint}
}
